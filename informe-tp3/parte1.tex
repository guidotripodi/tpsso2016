
\subsection{Algoritmo}
\begin{itemize}
 \item \textbf{Implementacion}
Nuestro algoritmo se implemento de la siguiente manera:\\

Chequeamos si tenemos un mensaje, en caso de tenerlo revisamos el tipo de TAG, para verificar
si el mismo es un ACK o un token de otro proceso que se encuentra en circulación en el anillo.\\ Una vez realizado dicha verificación, si el mismo es un \textbf{TAG$\_$OTORGADO} (utilizamos dicho tag para enviar y recibir token) procedemos a chequear los valores token[0] y token[1] como se solicito.\\
Una vez verificado si el token recibido es propio o ajeno y si soy lider o no procedemos a actualizarlo con los respectivos datos:\\

\begin{verbatim}
	Cuando se recibe un mensaje de eleccion de lider <i, cl>, se actua de la siguiente manera:
• Hay una eleccion de lıder dando vueltas. status:= no lıder.
• Si i==ID, la eleccion dio toda la vuelta al anillo, y cl es el lıder. Si cl>ID, significa
que el lıder esta mas adelante y no sabe que gano. El token debe seguir girando con
los valores <cl, cl>. Si cl==ID, soy el lıder y debo actualizar status a lıder.
• Si i!=ID, el token debe seguir girando. Si ID>cl, hay un nuevo candidato a lıder. Es
decir, cl:= ID.
\end{verbatim}

En caso de no ser lider, procedemos a enviar el token al próximo proceso del anillo, aquí quedamos en espera de un ACK del próximo pid, si pasado 2 tiempo no recibimos dicho ACK le enviamos el mismo token al próximo, así hasta que alguno de dichos procesos nos responda o peguemos la vuelta al anillo. (En la parte 2 de este informe aclararemos ciertos puntos referidos a este tema ya que dicha espera suele perjudicar ampliamente a la hora de que el token viaje por el anillo buscando la elección correcta del líder)\\

En cambio, si recibimos un \textbf{TAG$\_$ACK}, verificamos si estabamos en espera del mismo en caso afirmativo, dejamos en$\_$espera = 0 y actualizamos el valor de nuestro proximo pid actual.\\
Si recibimos un ACK el cual el origen no se encuentre dentro de los rangos pid+1 $<=$ origen $<=$ proximo levantaremos el mismo pero no cambiaremos el valor de espera.\\

\end{itemize}
