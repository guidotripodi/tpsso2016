
\subsection{Ejercicios}
\begin{itemize}
 \item 
\textbf{Ejercicio 3}
En segundo lugar, deberán implementar el servidor de backend multithreaded inspirándose en el código provisto y lo desarrollado en el punto anterior.
\end{itemize}

\subsection{Resultados y Conclusiones}


\subsubsection[Resolución Ejercicio 3]{Ejercicio 3}

\indent En este Ejercicio, se solicit\'{o} la implementación de un backend mutithreaded, para el desarrollo del mismo,
utilizamos la base del backend mono para la conexi\'{o}n con el servidor, el parseo de las fichas, tanto para la validez de las mismas
y tambi\'{e}n el env\'{\i}o de dimensiones del tablero de juego.\\

Utilizamos la función $atendedor\_de\_jugador$ la cual la convertimos en un thread para cada jugador. Esta función es llamada desde
la función main cuando las conexiones del socket entre el cliente-servidor son correctas. \\
Convertimos la función enunciada de la siguiente manera:\\
\begin{verbatim}
 pthread_create(&threads[i], NULL, &atendedor_de_jugador, &socketfd_cliente);
\end{verbatim}

En donde, threads es un arreglo de thread$\_$t y se le asigna uno a cada jugador y socketfd$\_$cliente es el atributo 
de cada thread. En este ejercicio como no vamos a necesitar el atributo todos los threads usan el mismo.

A continuación, mostraremos esta sección de código:\\
\begin{verbatim}
\*creacion de arreglo de threads *\

                      pthread_t threads[NUM_THREADS];
                      
\end{verbatim}

Luego, en la función $atendedor\_de\_jugador$ la cual recibe un thread$\_$data lo guardamos en un puntero a thread$\_$data llamado
my$\_$data y creamos un entero llamado socket$\_$fd el socket que nos viene como parametro.\\

Luego, basandonos en el backend mono, realizamos una implementación similar con la particularidad que, en el if donde se
consulta si el mensaje del jugador es una parte del barco, el barco terminado, una bomba o update.\\
En caso de ser una parte de barco, luego de parsear el casillero, y al chequear la validez de la ficha utilizamos nuestro read\_write\_lock y realizamos
la funcion $wlock()$. En caso de ser una ficha valida, realizamos un write lock y procedemos a escribir de la siguiente manera:\\
\begin{verbatim}
				(*rwlocks_tablero)[ficha.fila][ficha.columna].wlock();
				(*tablero_jugador)[ficha.fila][ficha.columna] = ficha.contenido;
				(*rwlocks_tablero)[ficha.fila][ficha.columna].wunlock();
                     
                     
\end{verbatim}
Para luego enviar la misma y terminar la jugada. En caso de que la validez de la ficha no sea correcta, dejamos de leer
y procedemos a escribir para quitar fichas y asi dejar de escribir.

Por consiguiente, en caso de que el mensaje sea una barco terminado se realiza un wlock para escribir la palabra y luego un wunlock.\\

El mismo se detalla a continuación:\\
\begin{verbatim}
			for (list<Casillero>::const_iterator casillero = barco_actual.begin(); casillero != barco_actual.end(); casillero++) {
				(*rwlocks_tablero)[casillero->fila][casillero->columna].wlock();
				(*tablero_jugador)[casillero->fila][casillero->columna] = casillero->contenido;
				(*rwlocks_tablero)[casillero->fila][casillero->columna].wunlock();
			}
			barco_actual.clear();       
\end{verbatim}

Luego, en caso de ser una bomba chequeamos si el casillero donde se coloco la bomba tiene un barco o una bomba, en caso de tener una bomba se envia el mensaje de que ya estaba golpeado como en el backend mono. Si había un barco, hacemos un wlock para escribir en el casillero bomba y liberamos el lock.\\

De esta manera, queda implementado nuestro backend multithreaded como fue solicitado.






