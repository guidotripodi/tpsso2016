
\subsection{Ejercicios}
\begin{itemize}
 \item 
\textbf{Ejercicio 2}
Deber\'{a}n, a su vez, implementar un test para su implementaci\'{o}n de Read-Write Locks (RWLockTest)
que involucre la creaci\'{o}n de varios threads lectores y escritores donde cada uno de ellos trate
de hacer un lock sobre un mismo recurso y se vea que no haya deadlocks ni inanici\'{o}n (archivo
backend-multi/RWLockTest.cpp).
\end{itemize}

\subsection{Resultados y Conclusiones}


\subsubsection[Resolución Ejercicio 2]{Ejercicio 2}

\indent Para mostrar esto creamos una funci\'{o}n que crea varios threads de escritura y lectura. Cada tipo de thread 
llama a una funci\'{o}n distinta e imprime por pantalla su ID y el valor de una variable, en caso de los threads de 
escritura estos indican el nuevo valor de la variable mientras que los de lectura imprimen el valor actual de la misma.\\
\indent Para ello se utilizan las siguientes variables y funciones:\\
\begin{itemize}
\item int NUM\_THREADS: Indica la cantidad de threads a crear.
\item int variable: Esta es el recurso que los threads van a manejar ( ya sea para leerlo o modificarlo).
\item RWLock para\_variable: Este lock se usa cuando un thread quiere manejar el recurso.
\item int array[NUM\_THREADS]: Un array de tama\~{n}o igual a cantidad de threads, Se usa para darle a cada thread un ID. 
La posici\'{o}n del array va a corresponder al thread creado.
\item pthread\_t threads[NUM\_THREADS]: Un array de tama\~{n}o igual a cantidad de threads a crear, se usa para crear los threads.
\item void *soy\_lector(void *p\_numero): Funci\'{o}n que va a correr el thread que lea el recurso.
\item void *soy\_escritor(void *p\_numero): Funci\'{o}n a correr por los threads que modifiquen el valor del recurso.
\end{itemize}

\begin{verbatim}
for (i = 0; i < NUM_THREADS; ++i){
  if(!(i % 2)){
    pthread_create(&threads[i], NULL, soy_escritor, &array[i]);
  }else{
    pthread_create(&threads[i], NULL, soy_lector, &array[i]);
  }
}
\end{verbatim}

Este ciclo es el encargado de crear todos los threads, si el valor de i es impar se crea un thread que va a correr la funci\'{o}n 
"soy\_escritor", si es par ejecuta "soy\_lector". \\
Para crear cada thread se le pasa a la funci\'{o}n un thread dentro del array "threads" y la funci\'{o}n a partir de donde va 
a correr( en este caso una de las 2 funciones ya mencionadas) y una posici\'{o}n del array "array" distinta a cada uno para 
poder identificarlos.\\

\begin{verbatim}
for (int j = 0; j < NUM_THREADS; ++j){
  pthread_join(threads[j], NULL);
}
pthread_exit(NULL);
return 0;
\end{verbatim}
Luego se hace un ciclo para esperar a que terminen todos los threads creados antes de finalizar el programa.\\

\begin{verbatim}
void *soy_lector(void *p_numero){
  int mi_numero = *((int *) p_numero);

  for (int j = 0; j < 5; ++j){
    para_variable.rlock();
    printf("Lector numero: %d  ", mi_numero);
    printf(" Leo valor %d\n", variable );
    para_variable.runlock();
  }
  pthread_exit(NULL);
  return NULL;
}
\end{verbatim}

La funci\'{o}n lector corre un ciclo 5 veces. Dentro del ciclo lo primero que hace es usar el lock mientras trabaja con la 
variable a usar por todos los threads. Luego imprime su ID que se le paso por par\'{a}metro a la funci\'{o}n y el valor 
actual de la variable y libera el lock. Luego de hacer esto 5 veces termina el thread.\\


\begin{verbatim}
para_variable.wlock();
variable++;
printf("Escritor numero: %d  ", mi_numero);
printf(" cambio valor %d\n", variable );
para_variable.wunlock();
\end{verbatim}

La funci\'{o}n "soy\_escritor" es casi igual al de la otra funci\'{o}n salvo porque lo primero que hace al hacer el lock es 
incrementar en 1 el valor de la variable. \\

\subsubsection{Conclusiones}

Con el c\'{o}digo que implementamos nos aseguramos que no pueda haber deadlock ni inanici\'{o}n. Vamos a pasar a explicar que 
soluci\'{o}n encontramos para cada uno. \\

\begin{itemize}
 \item Deadlock: Sabemos que para que haya deadlock hay 4 cosas fundamentales que tienen que cumplirse y a\'{u}n as\'{\i} no 
 siempre hay. Se observa en el c\'{o}digo explicado mas arriba que solo se usa "RWLock para\_variable" para proteger la secci\'{o}n 
 cr\'{\i}tica del c\'{o}digo por lo que no hay "Hold and wait" evitando as\'{\i} posible deadlock.\\
 Ahora vamos a ver que nuestro c\'{o}digo implementado para Read-Write Lock tampoco puede sufrir Deadlock. Igual que en el caso 
 anterior el c\'{o}digo implementado no puede sufrir de "Hold and wait". La \'{u}nica funci\'{o}n que pareciera poder sufrir de 
 esto es "wlock". Esta hace lock de "pthread\_mutex\_t turnstyle" y luego hace lock de "pthread\_mutex\_lock(\&readers\_mutex)". Si se 
 da el caso de que puede hacer lock de ambos en el paso siguiente hace unlock de "readers\_mutex" que luego combinado y usado 
 correctamente con la funci\'{o}n "wulock" se libera el otro mutex (turnstile) evitando el riesgo de "Hold and wait". \\
 En el otro caso que se puede dar, "wlock" se queda esperando el mutex "readers\_mutex". Si vemos cuales son las funciones que 
 usan el mutex "readers\_mutex" vemos que estas una vez tomado el mutex pueden correr su c\'{o}digo sin problemas finalizando 
 y liberando dicho mutex. Esto permite que "wlock" pueda m\'{a}s adelante seguir con su ejecuci\'{o}n sin problemas. Luego de 
 analizar ambos casos y sus posibles desenlaces podemos concluir que no hay riesgo de Deadlock usando Read-Write Lock. Luego de 
 analizar ambas implementaciones podemos confirmar que nuestro c\'{o}digo est\'{a} libre de Deadlock.
 \item Como vimos en el punto 1 nuestro c\'{o}digo esta libre de inanici\'{o}n.
 \end{itemize}