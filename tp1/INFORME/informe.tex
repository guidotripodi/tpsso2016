\documentclass[a4paper,10pt,twoside]{article}

\usepackage[top=1in, bottom=1in, left=1in, right=1in]{geometry}
\usepackage[utf8]{inputenc}
%\usepackage[spanish,es-ucroman,es-noquoting]{babel}
\usepackage{fancyhdr}
\usepackage{lastpage}
\usepackage{graphicx}



%%%%%%%%%% Configuración de Fancyhdr - Inicio %%%%%%%%%%
\pagestyle{fancy}
\thispagestyle{fancy}
\lhead{TP1, Sistemas Operativos}
\renewcommand{\footrulewidth}{0.4pt}
\cfoot{\thepage /\pageref{LastPage}}

\fancypagestyle{caratula} {
   \fancyhf{}
   \cfoot{\thepage /\pageref{LastPage}}
   \renewcommand{\headrulewidth}{0pt}
   \renewcommand{\footrulewidth}{0pt}
}

%%%%%%%%%% Macros de tikz - Fin %%%%%%%%%%


\begin{document}


%%%%%%%%%%%%%%%%%%%%%%%%%%%%%%%%%%%%%%%%%%%%%%%%%%%%%%%%%%%%%%%%%%%%%%%%%%%%%%%
%% Carátula                                                                  %%
%%%%%%%%%%%%%%%%%%%%%%%%%%%%%%%%%%%%%%%%%%%%%%%%%%%%%%%%%%%%%%%%%%%%%%%%%%%%%%%


\thispagestyle{caratula}

\begin{center}

\includegraphics[height=2cm]{DC.png} 
\hfill
\includegraphics[height=2cm]{UBA.jpg} 

\vspace{2cm}

Departamento de Computación,\\
Facultad de Ciencias Exactas y Naturales,\\
Universidad de Buenos Aires

\vspace{4cm}

\begin{Huge}
TP1 - Scheduling
\end{Huge}

\vspace{0.5cm}

\begin{Large}
Sistemas Operativos
\end{Large}

\vspace{1cm}

Primer Cuatrimestre de 2016

\vspace{4cm}

\vspace{0.5cm}

\begin{tabular}{|c|c|c|}
\hline
Apellido y Nombre & LU & E-mail\\
\hline
Tripodi, Guido			& 843/10 & guido.tripodi@hotmail.com\\
Sueiro, Diego           &  75/90 & dsueiro@gmail.com\\
Tripodi, Guido			& 843/10 & guido.tripodi@hotmail.com\\
\hline
\end{tabular}

\end{center}

\newpage


%%%%%%%%%%%%%%%%%%%%%%%%%%%%%%%%%%%%%%%%%%%%%%%%%%%%%%%%%%%%%%%%%%%%%%%%%%%%%%%
%% �?ndice                                                                    %%
%%%%%%%%%%%%%%%%%%%%%%%%%%%%%%%%%%%%%%%%%%%%%%%%%%%%%%%%%%%%%%%%%%%%%%%%%%%%%%%


\tableofcontents

\newpage


%%%%%%%%%%%%%%%%%%%%%%%%%%%%%%%%%%%%%%%%%%%%%%%%%%%%%%%%%%%%%%%%%%%%%%%%%%%%%%%
%% Introducción                                                              %%
%%%%%%%%%%%%%%%%%%%%%%%%%%%%%%%%%%%%%%%%%%%%%%%%%%%%%%%%%%%%%%%%%%%%%%%%%%%%%%%


\section{Introducción}

\indent \indent En este Trabajo Pr\'{a}ctico estudiaremos diversas implementaciones de algoritmos de scheduling. 
Haciendo uso de un simulador provisto por la c\'{a}tedra podremos representar el comportamiento de estos algoritmos. 
Implementaremos dos Round-Robin, uno que permite migraci\'{o}n de tareas entre n\'{u}cleos y otro que no y a trav\'{e}s de experimentaci\'{o}n intentaremos comparar ambos algoritmos. 
Asimismo, bas\'{a}ndonos en un scheduling del cual s\'{o}lo tenemos la versi\'{o}n ejecutable realizaremos una serie de experimentos para luego implementar el mismo.\\

%%%%%%%%%%%%%%%%%%%%%%%%%%%%%%%%%%%%%%%%%%%%%%%%%%%%%%%%%%%%%%%%%%%%%%%%%%%%%%%
%% Desarrollo                                                                %%
%%%%%%%%%%%%%%%%%%%%%%%%%%%%%%%%%%%%%%%%%%%%%%%%%%%%%%%%%%%%%%%%%%%%%%%%%%%%%%%

\newpage
\section{Desarrollo y Resultados}

\section{Parte I: Entendiendo el simulador simusched}


\subsection{Ejercicios}
\begin{itemize}
 \item \textbf{Ejercicio 1 } Programar un tipo de tarea TaskConsola, que simular\'{a} una tarea interactiva.
La tarea debe realizar n llamadas bloqueantes, cada una de una duraci\'{o}n al azar 1 entre bmin
y bmax (inclusive). La tarea debe recibir tres par\'{a}metros: n, bmin y bmax (en ese orden)
que ser\'{a}n interpretados como los tres elementos del vector de enteros que recibe la funci\'{o}n.
Explique la implementaci\'{o}n realizada y grafique un lote que utilice el nuevo tipo de tarea.
\item \textbf{Ejercicio 2} El grupo de competencia de Data Mining, reciente ganador de importante concurso internacional, esta
preparando el algoritmo para su próxima victoria. Para esto necesita utilizar fuertemente la CPU por 500 ciclos.
A su vez, el grupo usa la máquina como servidor remoto, utilizando 3 usuarios que realizan llamadas bloqueantes de 10, 20 y 30 
respectivamente y de una duración al azar de hasta 4 ciclos.
Escribir el lote de tareas que simule la situación del grupo. Ejecutar y graficar la simulación usando el algoritmo FCFS para 1 y 2 y 4 
núcleos con un cambio de contexto de 5 ciclos. Calcular la latencia de cada tarea en los dos gráficos. Explicar 
que desventaja tendría si debe mantener este algoritmo de scheduling y solo tiene disponible una computadora con un núcleo (haga
referencia a los gráficos y a los cálculos anteriores para justificar su explicación).

\item \textbf{Ejercicio 3} Programar un tipo de tarea TaskBatch que reciba dos par\'{a}metros: total cpu y
cant bloqueos. Una tarea de este tipo debera realizar cant bloqueos llamadas bloqueantes, en
momentos elegidos pseudoaleatoriamente. En cada tal ocasi\'{o}n, la tarea deber\'{a} permanecer
bloqueada durante exactamente dos (4) ciclos de reloj. \\
El tiempo de CPU total que utilice una
tarea TaskBatch deber\'{a} ser de total cpu ciclos de reloj (incluyendo el tiempo utilizado para
lanzar las llamadas bloqueantes; no as\'{i} el tiempo en que la tarea permanezca bloqueada).
Explique la implementaci\'{o}n realizada y grafique un lote que utilice 4 tareas TaskBatch con
par\'{a}metros diferentes y que corra con el scheduler FCFS.
\end{itemize}

\subsection{Resultados y Conclusiones}

\subsubsection[Resolución Ejercicio 1]{Ejercicio 1}

\indent Dada la simpleza del código, optamos por mostrar nuestra implementación, en vez de comentarlo\\ detalladamente.\\
\indent Realizamos un ciclo de i \textless \ params[0], donde utilizamos la función dada por la catedra, uso\_IO a la cual le pasamos
el pid correspondiente y un entero ciclos que es el valor random obtenido entre $bmin$ y $bmax$. Esa función uso\_IO simula una llamada bloqueante.
\begin{center}
 \begin{verbatim}
                     ciclos = rand() % (params[2] - params[1] + 1) + params[1];
 \end{verbatim}

\end{center}

\indent A continuaci\'{o}n, el c\'{o}digo mencionado:

\begin{verbatim}

                  void TaskConsola(int pid, vector<int> params) {
                       int i, ciclos;              
                       for (i = 0; i < params[0]; i++) {
                              ciclos = rand() % (params[2] - params[1] + 1) + params[1];  
                              uso_IO(pid, ciclos);
                       }
                  } 

\end{verbatim}

\indent Como experimentaci\'{o}n trabajamos con el siguiente lote:\\

\begin{verbatim}
                              TaskConsola 5 3 7
                              TaskConsola 2 3 3
                              TaskConsola 15 2 7
\end{verbatim}

De aqu\'{i}, obtuvimos los siguientes resultados:\\

\vspace*{0.3cm} \vspace*{0.3cm}
  \begin{center}
 \includegraphics[scale=0.5]{./Test/ej1.png}
 { $Gr$\'a$fico 1.1$ Scheduler FCFS - 1 core }
 \end{center}
  \vspace*{0.3cm}



\subsubsection[Resolución Ejercicio 2]{Ejercicio 2}


\indent ESTE HAY QUE HACERLO

\subsubsection[Resolución Ejercicio 3]{Ejercicio 3}

EN ESTE EJERCICIO HAY Q CAMBIAR EL USO IO DE 1 A 4 Y CAMBIAR EL EXPERIMENTO

\indent Al igual que con la tarea TaskConsola, mencionaremos nuestro implementación y por consiguiente  
explicaremos ciertos puntos de la misma.\\
 \begin{verbatim}
                       void TaskBatch(int pid, vector<int> params) {
                            int total_cpu = params[0];
                            int cant_bloqueos = params[1];
                            vector<bool> uso = vector<bool>(total_cpu);
                            for(int i=0;i<(int)uso.size();i++) 
                               uso[i] = false;
                               for(int i=0;i<cant_bloqueos;i++) {
                                  int j = rand()%(uso.size());
                                  if(!uso[j])
                                     uso[j] = true;
                                  else
                                     i--; 
                               }
                               for(int i=0;i<(int)uso.size();i++) {
                                  if( uso[i] )
                                     uso_IO(pid,1); 
                                  else
                                     uso_CPU(pid, 1); 
                               }
                       }
 \end{verbatim}

 \indent Para este tipo de tarea, creamos un vector de tamaño igual a $total\_cpu$ el cual contiene valores booleanos, 
 si el valor en el \'{\i}ndice del vector es true este corresponder\'a a la funci\'{o}n uso\_IO, caso contrario uso\_CPU.\\
 Luego, utilizaremos un ciclo que irá desde 0 hasta el tamaño del vector y dependiendo el valor booleano, usará la funciones
 dadas por la catedra uso\_IO o uso\_CPU.\\
 
 \indent El experimento realizado para este nuevo tipo de tarea fue el siguiente:\\
 
 Con un lote de tareas:\\
 
 \begin{verbatim}
                           TaskBatch 10 3
                           TaskBatch 5 4
                           TaskBatch 8 1
 \end{verbatim}

 Obtuvimos el siguiente diagrama:\\
 
 \vspace*{0.3cm} \vspace*{0.3cm}
  \begin{center}
 \includegraphics[scale=0.5]{./Test/ej3.png}
 { $Gr$\'a$fico 3.1$ Scheduler FCFS - 1 core }
 \end{center}
  \vspace*{0.3cm}
 

\newpage
%\subsection{Filtro Miniature}

\section{Parte II: Extendiendo el simulador con nuevos schedulers}


\subsection{Ejercicios}
\begin{itemize}
 \item 
\textbf{Ejercicio 2}
Deber\'{a}n, a su vez, implementar un test para su implementaci\'{o}n de Read-Write Locks (RWLockTest)
que involucre la creaci\'{o}n de varios threads lectores y escritores donde cada uno de ellos trate
de hacer un lock sobre un mismo recurso y se vea que no haya deadlocks ni inanici\'{o}n (archivo
backend-multi/RWLockTest.cpp).
\end{itemize}

\subsection{Resultados y Conclusiones}


\subsubsection[Resolución Ejercicio 2]{Ejercicio 2}

\indent Para mostrar esto creamos una funci\'{o}n que crea varios threads de escritura y lectura. Cada tipo de thread 
llama a una funci\'{o}n distinta e imprime por pantalla su ID y el valor de una variable, en caso de los threads de 
escritura estos indican el nuevo valor de la variable mientras que los de lectura imprimen el valor actual de la misma.\\
\indent Para ello se utilizan las siguientes variables y funciones:\\
\begin{itemize}
\item int NUM\_THREADS: Indica la cantidad de threads a crear.
\item int variable: Esta es el recurso que los threads van a manejar ( ya sea para leerlo o modificarlo).
\item RWLock para\_variable: Este lock se usa cuando un thread quiere manejar el recurso.
\item int array[NUM\_THREADS]: Un array de tama\~{n}o igual a cantidad de threads, Se usa para darle a cada thread un ID. 
La posici\'{o}n del array va a corresponder al thread creado.
\item pthread\_t threads[NUM\_THREADS]: Un array de tama\~{n}o igual a cantidad de threads a crear, se usa para crear los threads.
\item void *soy\_lector(void *p\_numero): Funci\'{o}n que va a correr el thread que lea el recurso.
\item void *soy\_escritor(void *p\_numero): Funci\'{o}n a correr por los threads que modifiquen el valor del recurso.
\end{itemize}

\begin{verbatim}
for (i = 0; i < NUM_THREADS; ++i){
  if(!(i % 2)){
    pthread_create(&threads[i], NULL, soy_escritor, &array[i]);
  }else{
    pthread_create(&threads[i], NULL, soy_lector, &array[i]);
  }
}
\end{verbatim}

Este ciclo es el encargado de crear todos los threads, si el valor de i es impar se crea un thread que va a correr la funci\'{o}n 
"soy\_escritor", si es par ejecuta "soy\_lector". \\
Para crear cada thread se le pasa a la funci\'{o}n un thread dentro del array "threads" y la funci\'{o}n a partir de donde va 
a correr( en este caso una de las 2 funciones ya mencionadas) y una posici\'{o}n del array "array" distinta a cada uno para 
poder identificarlos.\\

\begin{verbatim}
for (int j = 0; j < NUM_THREADS; ++j){
  pthread_join(threads[j], NULL);
}
pthread_exit(NULL);
return 0;
\end{verbatim}
Luego se hace un ciclo para esperar a que terminen todos los threads creados antes de finalizar el programa.\\

\begin{verbatim}
void *soy_lector(void *p_numero){
  int mi_numero = *((int *) p_numero);

  for (int j = 0; j < 5; ++j){
    para_variable.rlock();
    printf("Lector numero: %d  ", mi_numero);
    printf(" Leo valor %d\n", variable );
    para_variable.runlock();
  }
  pthread_exit(NULL);
  return NULL;
}
\end{verbatim}

La funci\'{o}n lector corre un ciclo 5 veces. Dentro del ciclo lo primero que hace es usar el lock mientras trabaja con la 
variable a usar por todos los threads. Luego imprime su ID que se le paso por par\'{a}metro a la funci\'{o}n y el valor 
actual de la variable y libera el lock. Luego de hacer esto 5 veces termina el thread.\\


\begin{verbatim}
para_variable.wlock();
variable++;
printf("Escritor numero: %d  ", mi_numero);
printf(" cambio valor %d\n", variable );
para_variable.wunlock();
\end{verbatim}

La funci\'{o}n "soy\_escritor" es casi igual al de la otra funci\'{o}n salvo porque lo primero que hace al hacer el lock es 
incrementar en 1 el valor de la variable. \\

\subsubsection{Conclusiones}

Con el c\'{o}digo que implementamos nos aseguramos que no pueda haber deadlock ni inanici\'{o}n. Vamos a pasar a explicar que 
soluci\'{o}n encontramos para cada uno. \\

\begin{itemize}
 \item Deadlock: Sabemos que para que haya deadlock hay 4 cosas fundamentales que tienen que cumplirse y a\'{u}n as\'{\i} no 
 siempre hay. Se observa en el c\'{o}digo explicado mas arriba que solo se usa "RWLock para\_variable" para proteger la secci\'{o}n 
 cr\'{\i}tica del c\'{o}digo por lo que no hay "Hold and wait" evitando as\'{\i} posible deadlock.\\
 Ahora vamos a ver que nuestro c\'{o}digo implementado para Read-Write Lock tampoco puede sufrir Deadlock. Igual que en el caso 
 anterior el c\'{o}digo implementado no puede sufrir de "Hold and wait". La \'{u}nica funci\'{o}n que pareciera poder sufrir de 
 esto es "wlock". Esta hace lock de "pthread\_mutex\_t turnstyle" y luego hace lock de "pthread\_mutex\_lock(\&readers\_mutex)". Si se 
 da el caso de que puede hacer lock de ambos en el paso siguiente hace unlock de "readers\_mutex" que luego combinado y usado 
 correctamente con la funci\'{o}n "wulock" se libera el otro mutex (turnstile) evitando el riesgo de "Hold and wait". \\
 En el otro caso que se puede dar, "wlock" se queda esperando el mutex "readers\_mutex". Si vemos cuales son las funciones que 
 usan el mutex "readers\_mutex" vemos que estas una vez tomado el mutex pueden correr su c\'{o}digo sin problemas finalizando 
 y liberando dicho mutex. Esto permite que "wlock" pueda m\'{a}s adelante seguir con su ejecuci\'{o}n sin problemas. Luego de 
 analizar ambos casos y sus posibles desenlaces podemos concluir que no hay riesgo de Deadlock usando Read-Write Lock. Luego de 
 analizar ambas implementaciones podemos confirmar que nuestro c\'{o}digo est\'{a} libre de Deadlock.
 \item Como vimos en el punto 1 nuestro c\'{o}digo esta libre de inanici\'{o}n.
 \end{itemize}
\newpage

\section{Aclaraciones y Bibliografia}


\subsection{Ejercicios}
\begin{itemize}
 \item 
\textbf{Ejercicio 3}
En segundo lugar, deberán implementar el servidor de backend multithreaded inspirándose en el código provisto y lo desarrollado en el punto anterior.
\end{itemize}

\subsection{Resultados y Conclusiones}


\subsubsection[Resolución Ejercicio 3]{Ejercicio 3}

\indent En este Ejercicio, se solicit\'{o} la implementación de un backend mutithreaded, para el desarrollo del mismo,
utilizamos la base del backend mono para la conexi\'{o}n con el servidor, el parseo de las fichas, tanto para la validez de las mismas
y tambi\'{e}n el env\'{\i}o de dimensiones del tablero de juego.\\

Utilizamos la función $atendedor\_de\_jugador$ la cual la convertimos en un thread para cada jugador. Esta función es llamada desde
la función main cuando las conexiones del socket entre el cliente-servidor son correctas. \\
Convertimos la función enunciada de la siguiente manera:\\
\begin{verbatim}
 pthread_create(&threads[i], NULL, &atendedor_de_jugador, &socketfd_cliente);
\end{verbatim}

En donde, threads es un arreglo de thread$\_$t y se le asigna uno a cada jugador y socketfd$\_$cliente es el atributo 
de cada thread. En este ejercicio como no vamos a necesitar el atributo todos los threads usan el mismo.

A continuación, mostraremos esta sección de código:\\
\begin{verbatim}
\*creacion de arreglo de threads *\

                      pthread_t threads[NUM_THREADS];
                      
\end{verbatim}

Luego, en la función $atendedor\_de\_jugador$ la cual recibe un thread$\_$data lo guardamos en un puntero a thread$\_$data llamado
my$\_$data y creamos un entero llamado socket$\_$fd el socket que nos viene como parametro.\\

Luego, basandonos en el backend mono, realizamos una implementación similar con la particularidad que, en el if donde se
consulta si el mensaje del jugador es una parte del barco, el barco terminado, una bomba o update.\\
En caso de ser una parte de barco, luego de parsear el casillero, y al chequear la validez de la ficha utilizamos nuestro read\_write\_lock y realizamos
la funcion $wlock()$. En caso de ser una ficha valida, realizamos un write lock y procedemos a escribir de la siguiente manera:\\
\begin{verbatim}
				(*rwlocks_tablero)[ficha.fila][ficha.columna].wlock();
				(*tablero_jugador)[ficha.fila][ficha.columna] = ficha.contenido;
				(*rwlocks_tablero)[ficha.fila][ficha.columna].wunlock();
                     
                     
\end{verbatim}
Para luego enviar la misma y terminar la jugada. En caso de que la validez de la ficha no sea correcta, dejamos de leer
y procedemos a escribir para quitar fichas y asi dejar de escribir.

Por consiguiente, en caso de que el mensaje sea una barco terminado se realiza un wlock para escribir la palabra y luego un wunlock.\\

El mismo se detalla a continuación:\\
\begin{verbatim}
			for (list<Casillero>::const_iterator casillero = barco_actual.begin(); casillero != barco_actual.end(); casillero++) {
				(*rwlocks_tablero)[casillero->fila][casillero->columna].wlock();
				(*tablero_jugador)[casillero->fila][casillero->columna] = casillero->contenido;
				(*rwlocks_tablero)[casillero->fila][casillero->columna].wunlock();
			}
			barco_actual.clear();       
\end{verbatim}

Luego, en caso de ser una bomba chequeamos si el casillero donde se coloco la bomba tiene un barco o una bomba, en caso de tener una bomba se envia el mensaje de que ya estaba golpeado como en el backend mono. Si había un barco, hacemos un wlock para escribir en el casillero bomba y liberamos el lock.\\

De esta manera, queda implementado nuestro backend multithreaded como fue solicitado.







\newpage


\end{document}
