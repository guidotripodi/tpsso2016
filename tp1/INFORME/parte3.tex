
\subsection{Makefile}

Nuestro Makefile fue implementado para poder realizar la experimentaci\'on por ejercicio, elimin\'{a}ndolos individualmente y/o todos juntos.\\
Los nombres son:\\
\begin{itemize}
 \item ejercicio1: make ej1 y make cleanej1.
 \item ejercicio2: make ej2 y make cleanej2.
 \item ejercicio3: make ej3 y make cleanej3.
 \item ejercicio4: make ej4 y make cleanej4.
 \item ejercicio5: make ej5 y make cleanej5.
 \item ejercicio6: make ej6 y make cleanej6.
 \item ejercicio7: make ej7 y make cleanej7.
 \item ejercicio8: make ej8 y make cleanej8.
\end{itemize}

Al hacer make, se crear\'a una carpeta con el ejercicio y dentro de la misma se encontrara/n el/los gr\'afico/s
Para eliminar todos los gr\'aficos y directorios hacer make cleanTodosLosEjercicios, igualmente este make clean est\'a implementado
para eliminar carpetas y/o archivos si estos existiesen o no.\\

\subsection{Bibliografia}

\begin{itemize}
 \item C\'atedra de Sistemas Operativos - Clases te\'oricas y pr\'acticas (1º Cuatrimestre 2016)
 \end{itemize}
